\chapter*{ABSTRACT}
RFId systems allow remote recognition of items by
means of radio communications.
A tag is coupled to an object and it can record information
such as personal data, photos and other information.
The tag can be inserted into objects and coated with different
materials for the various usages. Moreover, it can be customized
with prints, images and bar codes.
A reader interrogates the tags to obtain the information of
interest.
Before using an RFId system for a safety application, one has
to know and understand the best way it operates.
RFId systems allow innovative solutions to achieve some of
the health and safety requirements, in particular: in machinery
applications, where RFId systems can be used as additional
safety device, as interlocking device, as additional PPE; in
workplaces, where they can be used as an access key and for
localization of workers; as a safety inventory; for environmental
parameters recording; as a solution for installations; in the
medical sector, for the identification and localization of
equipment, patients and workers, for the tracking of surgical
instruments, for monitoring biological parameters or the state of
the treatment; and also for the improvement of the accuracy of
excavations in the subsoil.






