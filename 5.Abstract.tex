\chapter*{ABSTRACT}
%\paragraph{}
%\justifying
%\nointent
%\linespread{2}
Online Social Networks (ONS) have become integral part of our daily life. Photo sharing is one of the most important features of Online Social Networks... Unfortunately which may be used for purposes we never expect. To prevent possible privacy leakage of a group photo, we design a mechanism in which each individual can participate in the decision making on the photo posting. . For this purpose, we need an efficient facial recognition (FR) system that cans recognize everyone in the photo. We are using  Haar cascade classifier for  face detection and CBIR (content based image retrieval) algorithm to train individual’s images and for face recognition. To get enough training sample is actually little difficult task, so FR engine may be unsuccessful to identify the faces of each individual in a group photo. To avoid this we are using an efficient CBIR algorithm. Once the faces are identified from the group photo then acceptance notifications are sending automatically to each identified persons within the close friend circle. The photo will be posted if all the people within the friend circle are accepting the notification; the photo will not be posted if any one of them rejects the notification. We expect that our proposed scheme would be very useful in protecting users’ privacy in photo/image sharing over online social networks.







